%
% desc05_21.tex
% (max, +) system description
% automatically generated by phoebe ver.1.0 on 2019-10-13 16:46:20 
% Copyright (c) 2017-2019 Jarosław Stańczyk <j.stanczyk@hotmail.com>
%

\documentclass[11pt, a4paper, fleqn]{article}

\usepackage{amsmath}
\usepackage{currfile}
\usepackage{graphicx}

\begin{document}

\noindent
\textbf{(max, +) description of} \texttt{\currfilebase} \\
automatically generated by phoebe ver.1.0 on 2019-10-13 16:46:20 

\begin{align}\begin{split}
\mathbf{x}(k+1) & \, = \; \mathbf{A}_{0}\mathbf{x}(k+1) \oplus \mathbf{A}_{1}\mathbf{x}(k) \oplus \mathbf{B}_{1}\mathbf{u}(k)\\
& \, = \; \mathbf{Ax}(k) \oplus \mathbf{Bu}(k)\\
\mathbf{y}(k) & \, = \; \mathbf{Cx}(k)
\end{split}\end{align}

% vector u(k)
\begin{equation*}
\mathbf{u}(k) = 
\left[\begin{array}{*{20}c}
  u_1(k) \\
  u_2(k) \\
  u_3(k) \\
  u_4(k) \\
  u_5(k) \\
  u_6(k) \\
  u_7(k) \\
  u_8(k) \\
\end{array}\right]
\end{equation*}
% vector x(k)
\begin{equation*}
\mathbf{x}(k) = 
\left[\begin{array}{*{20}c}
  x_1(k) \\
  x_2(k) \\
  x_3(k) \\
  x_4(k) \\
  x_5(k) \\
  x_6(k) \\
\end{array}\right]
\end{equation*}
% vector y(k)
\begin{equation*}
\mathbf{y}(k) = 
\left[\begin{array}{*{20}c}
  y_1(k) \\
\end{array}\right]
\end{equation*}
\noindent\\
times:\\
$d_1 = 43$, $d_2 = 20$, $d_3 = 15$, $d_4 = 6$, $d_7 = 21$, $d_8 = 25$.\\
\\
matrices:
% matrix A_{0}
\begin{equation*}
\mathbf{A}_{0} = 
\left[\begin{array}{ cccccc }
\varepsilon	&\varepsilon	&\varepsilon	&\varepsilon	&\varepsilon	&\varepsilon\\
\varepsilon	&\varepsilon	&\varepsilon	&\varepsilon	&\varepsilon	&\varepsilon\\
\varepsilon	&\varepsilon	&\varepsilon	&\varepsilon	&\varepsilon	&\varepsilon\\
\varepsilon	&d_2	&\varepsilon	&\varepsilon	&\varepsilon	&\varepsilon\\
d_1	&\varepsilon	&\varepsilon	&\varepsilon	&\varepsilon	&d_8\\
\varepsilon	&\varepsilon	&d_3	&d_4	&\varepsilon	&\varepsilon\\
\end{array}\right]
\end{equation*}

% matrix A_{1}
\begin{equation*}
\mathbf{A}_{1} = 
\left[\begin{array}{ cccccc }
d_1	&\varepsilon	&\varepsilon	&\varepsilon	&\varepsilon	&\varepsilon\\
\varepsilon	&d_2	&\varepsilon	&\varepsilon	&\varepsilon	&\varepsilon\\
\varepsilon	&\varepsilon	&d_3	&\varepsilon	&\varepsilon	&\varepsilon\\
\varepsilon	&\varepsilon	&\varepsilon	&d_4	&\varepsilon	&\varepsilon\\
\varepsilon	&\varepsilon	&\varepsilon	&\varepsilon	&d_7	&\varepsilon\\
\varepsilon	&\varepsilon	&\varepsilon	&\varepsilon	&\varepsilon	&d_8\\
\end{array}\right]
\end{equation*}

% matrix B_{1}
\begin{equation*}
\mathbf{B}_{1} = 
\left[\begin{array}{ cccccccc }
0	&0	&\varepsilon	&\varepsilon	&\varepsilon	&\varepsilon	&\varepsilon	&\varepsilon\\
\varepsilon	&\varepsilon	&0	&0	&\varepsilon	&\varepsilon	&\varepsilon	&\varepsilon\\
\varepsilon	&\varepsilon	&\varepsilon	&\varepsilon	&\varepsilon	&0	&0	&\varepsilon\\
\varepsilon	&\varepsilon	&\varepsilon	&\varepsilon	&0	&\varepsilon	&\varepsilon	&\varepsilon\\
\varepsilon	&\varepsilon	&\varepsilon	&\varepsilon	&\varepsilon	&\varepsilon	&\varepsilon	&\varepsilon\\
\varepsilon	&\varepsilon	&\varepsilon	&\varepsilon	&\varepsilon	&\varepsilon	&\varepsilon	&0\\
\end{array}\right]
\end{equation*}

% matrix C_{{}}
\begin{equation*}
\mathbf{C}_{{}} = 
\left[\begin{array}{ cccccc }
\varepsilon	&\varepsilon	&\varepsilon	&\varepsilon	&d_7	&\varepsilon\\
\end{array}\right]
\end{equation*}


\end{document}

% eof

